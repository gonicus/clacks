\documentclass[a4paper,oneside,11pt]{scrartcl}
\usepackage[left=2cm,right=2cm,top=2.0cm,bottom=2.5cm,includeheadfoot]{geometry}
\usepackage[utf8]{inputenc}  % Replace "utf8" by "latin1" if your editor sucks.
\usepackage{fancyhdr}        % Better header and footer support.
\usepackage{titlesec}        % Alternative section titles.
\usepackage{amssymb}         % Math support.
\usepackage{amsmath}         % Math declarations.

% GraphViz support
\usepackage{graphicx}
\usepackage[x11names, rgb]{xcolor}
\usepackage{tikz}            % Drawing shapes.
\usepackage{caption}         % Captions for graphviz figures.
\usetikzlibrary{snakes,arrows,shapes}
\newcommand{\mygraph}[2]{%
  \vspace{1em}
  \includegraphics[width=15cm]{figures/#1.png}
  \captionof{figure}{#2}
  \vspace{1em}
}

% Variables.
% This file is automatically generated; do not edit
\newcommand{\productversion}{0.3.0 }

\newcommand{\productname}{Spiff Signal}
\newcommand{\product}{{\it \productname} }

% Make references clickable.
\usepackage[colorlinks,hyperindex]{hyperref}
\hypersetup{%
  pdftitle    = {\productname\ Version \productversion},
  pdfkeywords = {spiff signal},
  pdfauthor   = {Samuel Abels},
  colorlinks  = true,
  %linkcolor   = blue,
}

% Initialize headers and footers.
\pagestyle{fancy}            % Use fancyhdr to render page headers/footers.
\fancyhf{}                   % Clear out old header/footer definition.

% Header
%\fancyhead[C]{\bfseries \productname}
\fancyhead[L]{\leftmark}
\fancyhead[R]{\MakeUppercase{\rightmark}}
\renewcommand{\headrulewidth}{0.5pt}

% Footer
\fancyfoot[C]{Page \thepage}
\renewcommand{\footrulewidth}{0.5pt}

% Enumerate using letters.
\renewcommand{\labelenumi}{\alph{enumi})}

% Set source code options.
\usepackage{listings}
\lstset{language=python}
\lstset{commentstyle=\textit}
\lstset{showstringspaces=false}
\lstset{aboveskip=.1in,belowskip=.1in,xleftmargin=2em,basewidth=5pt}

% Do not indent paragraphs.
\parindent=0em

% Preformatted, indented text.
\usepackage{verbatim}
\makeatletter
\newenvironment{indentverb}
  {\def\verbatim@processline{%
  \hspace*{2em}\the\verbatim@line\par}%
  \verbatim}
  {\endverbatim}
\makeatother

% Title
\title{\productname\ Release \productversion\\
User Documentation\\
\vspace{5 mm}
\large Signal/event mechanism for Python}

% Hint boxes.
\usepackage{color}
\definecolor{nb}{gray}{.90}
\newcommand{\hint}[1]{
  \begin{center}
  \colorbox{nb}{
    \begin{tabular}{ll}
      \Large ! &
      \begin{minipage}{.92\linewidth}{
        \vspace{2mm}
        \sf #1
        \vspace{2mm}
      }\end{minipage}
    \end{tabular}
  }
  \end{center}
}
     % Import common styles.
\usepackage{german}   % German language, old orthography.
\fancyfoot[C]{Seite \thepage}
\title{\productname\ Version \productversion\\
Benutzerhandbuch\\
\vspace{5 mm}
\large Signal-/Event-Mechanismus für Python}
\author{Samuel Abels}

\begin{document}
\maketitle
\tableofcontents

\newpage
\section{Einführung}
\subsection{Wozu \productname?}

\product stellt einen einfach Signal-/Event-Mechanismus für Python zur 
Verfügung.

\subsection{Rechtliches}

\product und das vorliegende Benutzerhandbuch werden unter den Bedingungen 
der GNU GPL (General Public License) Version 2 zur Verfügung gestellt. Eine 
Kopie der GPL sollten sie zusammen mit \product erhalten haben; falls nicht, 
so können sie diese unter der folgenden Adresse einsehen:

\vspace{1em}
\url{http://www.gnu.org/licenses/gpl-2.0.txt}
\vspace{1em}

Sollte diese Lizenz ihren Anforderungen nicht genügen, können sie sich 
gerne unter Angabe der Gründe über die im nächsten Abschnitt 
angegebenen Kontaktmöglichkeiten mit uns in Verbindung setzen - vielleicht 
lässt sich eine Lösung herbeiführen, die beide Seiten zufrieden stellt.


\subsection{Kontaktinformation \& Feedback}

Sollten sie Verbesserungsvorschläge oder Korrekturen für \product oder 
die \product-Dokumentation haben, so werden diese dankbar entgegengenommen.
Folgende Kontaktmöglichkeiten bieten wir an: \\

\begin{tabular}{ll}
{\bf Google Groups:} & http://groups.google.com/group/spiff-devel/ \\
{\bf Bug tracker:}   & http://code.google.com/p/spiff-workflow/issues/list \\
{\bf Phone:}         & +49 176 830 40288 \\
{\bf Jabber:}        & knipknap@jabber.org
\end{tabular}


\newpage
\section{Schnellüberblick}

\product stell nur eine einzige Klasse zur Verfügung: {\it Trackable} 
implementiert die Schnittstelle um Signale zu senden und zu empfangen.
Ein Objekt, das Signale senden möchte, muss lediglich von {\it Trackable} 
erben und kann dann die {\it signal\_emit}-Methode verwenden, wie in folgendem 
Code zu sehen:

\begin{lstlisting}
from SpiffSignal import Trackable

class WatchMe(Trackable):
    def __init__(self):
        Trackable.__init__(self)
    
    def do_something(self):
        self.signal_emit('did-something', 'hello world')
\end{lstlisting}

Um das Signal zu empfangen kann ein Abonnent sich dann mittels 
{\it signal\_connect} beim Sender anmelden:

\begin{lstlisting}
def my_callback(arg):
    print arg

foo = WatchMe()
foo.signal_connect('did-something', my_callback)
foo.do_something()
\end{lstlisting}

Es wird also beim Senden des {\it did\_something}-Signals die Funktion 
{\it my\_callback} aufgerufen.

\product stellt weitere Methoden zur Verfügung um bestehende Verbindungen 
vom Sender wieder abzumelden und um weitere Informationen zu ermitteln. 
Eine vollständige Liste aller von {\it Trackable} zur Verfügung 
gestellten Methoden finden sie in unserer API-Dokumentation.
\end{document}
