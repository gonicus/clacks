\chapter{Messaging mit AMQP}

\section{Was ist AMQP?}

AMQP steht für Advanced Message Queueing Protocol und stellt eine Middleware
für nachrichtenorientierte Anwendungen auf dem Applicationslayer des OSI-Modells
dar. Es kümmert sich um die Zustellung, Weiterleitung und Zwischenspeicherung von
Nachrichten, sowie die Zuverlässigkeit des Austausches sowie Sicherheit.

AMQP ist ein binäres Protokoll welches die Interoperabilität zwischen verschiedenen
Plattformen gewährleistet. Form und Grösse von Daten der einzelnen Nachrichten
spielen keine Rolle.

Eine AMQP Infrastruktur besteht aus mindestens einem Broker - das ist die
Instanz auf der die Nachricthen-Queues zusammenlaufen, sowie einer (TODO) beliebigen
Anzahl von Clients. Diese Clients können Nachrichten publizieren und/oder
empfangen. Im folgenden werden die Empfänger als Consumer bezeichnet, die
publizierenden Clients als Producer.


\section{Queues}

Grundlage für den Nachrichtenverkehr sind Queues. Die Nachrichten in Queues
werden im Speicher oder auf der Festplatte vorgehalten und auf Wunsch in der eingespeisten
Reihenfolge an die Consumer weitergeleitet. Queues sind damit Nachrichtenspeicher
sowie Verteilungseinheit in einem. Nachrichtenqueues sind voneinander
unabhängig und können verschiedene Eigenschaften besitzen: privat oder öffentlich,
permanent oder temporär, beständig oder flüchtig. Durch die Eigenschaften
einer Queue wird bestimmt, ob man z.B. eine store-and-forward, pub-sub oder
wie auch immer geartete Queue bereitstellen möchte.

Würde man eine Analogie zu SMTP anstrengen, so sind die Queues wie Postfächer
zu sehen. Der Absender der Mail ein Producer, ein Mail-Filter ein Consumer.

Queues lassen sich unter anderem so konfigurieren, dass one to one, one to any,
point to point sowie publish and subscribe realisierbar sind. Für einen
Teilnehmer im AMQP-Netzwerk kann es z.B. interessant sein wenn die Festplatte
eines angeschlossenen Systems kurz davor ist vollzulaufen. In diesem Fall
ist eine Status-Queue anzulegen, welche von den angeschlossenen Systemen
mit Status-Information befüllt wird. Consumer können dann darauf reagieren
und etwa eine Festplatten-Aufräum-Aktion starten oder einen Administrator
per SMS benachrichtigen.

Queues können persistent angelegt werden. Queues können über verschiedene
Broker in einem Cluster gespiegelt werden.


\section{Exchange}

Ein Exchange ist die Instanz im AMQP-Konzept, welche Queues über diverse
Kriterien miteinander verbinden kann und damit für das Routing von
Nachrichten zuständig ist. Diese Verbindung als Binding bezeichnet.

AMQP bringt bereits eine Anzahl von verschiedenen Exchanges mit. So lässt
sich z.B. anhand eines Routing-Keys oder über XQuery eine Zuordnung
zu diversen Nachrichtenqueues vornehmen.

Auf deren Verwendung wird Kapitel [TODO] Kommunikation genauer eingegangen.


\section{Nachrichten}

Die versendeten Informationen bestehen aus dem AMQP-Header und dem
Nachrichten-Body. Grundsätzlich ist der Body im AMQP-Konzept nicht
von Bedeutung: was transportiert wird spielt keine Rolle.

Sämtliche Konfiguration über das \glqq wie wird die Nachricht weitergeleitet\grqq geschieht
über den Header.


\section{Verfügbare Broker}

Einer der grössten Nachteile von AMQP ist, dass verschiedene Spezifikationen
in unterschiedlichen Entwicklungsständen vorliegen, die grösstenteils nicht
miteinander kompatibel sind. Bis zur Version 1.0 wird uns dieses Problem
wahrscheinlich noch erhalten bleiben.

Es existieren unterschiedliche frei verfügbare sowie kommerzielle Broker. Hier
ein Auszug:

\begin{itemize}
\item OpenAMQ

  Dieser freie Broker wurde von den ursrpünglichen Schöpfern des AMQP in C entwickelt und
  unterstützt mit zyre interesante Features wie RestMS. Leider wird OpenAMQ
  eingestellt und dem ebenso von iMatix entwickelten 0mq der Vortritt gelassen.

  OpenAMQ unterstützt nur ältere Spezifikationen von AMQP.

  Dieses Projekt bringt keinerlei Zugriffskontrolle und Authentifizierung mit
  und scheidet aufgrund des zusätzlichen Aufwandes an dieser Stelle aus.

\item RabitMQ

  Ist ein freier, in Java programmierter Broker. Er wird von der Firma
  Rabbit Technologies Ltd gepflegt.

  RabitMQ unterstützt nur ältere Spezifikationen von AMQP.

  Wir möchten gerne eine Java-Abhängigkeit vermeiden. Dieser Broker fällt
  damit aus.

\item QPID

  Dieses Apache-Projekt bietet einen C- und Java-basierten Broker mit der
  jeweils aktuellsten Version der AMQP-Spezifikation. Er unterstützt eine
  SASL-basierte Authentifizierung, sowie eine Berechtigungsvergabe auf
  einzelne Queues.

  QPID unterstützt zusätzlich eine inhaltbasierte Nachrichtenfilterung über XQuery.


\item Red Hat Enterprise MRG

  Dies ist eine kommerzielle Version von QPID.
\end{itemize}


Im Falle von GOsa ist QPID aufgrund der Variabilität, XML-Features und der SASL-
Anbindung vorzuziehen. QPID bereitet durch seine Messaging API bereits auf
kommende AMQP-Versionen vor und erleichtert künftige Entwicklungen. 


\section{Authentifizierung}

Der verwendete Broker unterstützt eine SASL-basierte Authentifizierung und
eine Dateibasierte Authorisierung bezüglich des Zugriffs auf die Qeueus. Die
Authorisierung ist allerdings nur für sehr grundlegende Aktionen denkbar, da
sie nicht dynamisch ist und damit keine sinnvolle Verwaltung von Gruppen und
Benutzern zulässt. Zudem ist von einer Ausweitung des Berechtigunskonzeptes auf
Benutzer aus Performancegründen abzusehen.

AMQP versendet auf Wunsch die Authentifizierungsinformation des Benutzers
mit. Damit lässt sich auf Seiten des GOsa-Dienstes eine einfache Authorisierung
des Nutzers durchführen.

